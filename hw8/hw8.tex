\documentclass[12pt]{article}

\usepackage[pdftex]{graphicx}
\usepackage{cancel}
\usepackage[margin=4cm]{geometry}
\usepackage[hidelinks]{hyperref}
\usepackage{fancyhdr}
\usepackage{amsmath}
\usepackage{amsfonts}
\usepackage{dirtytalk}
\usepackage{parskip}

\newcommand\tab[1][1cm]{\hspace*{#1}}
\newcommand{\HRule}{\rule{\linewidth}{0.5mm}}
\newcommand{\course}{COMS 474}

\setcounter{secnumdepth}{0} % Disable section/subsection numbering
\hyphenpenalty 10000 % Prevent words from being broken over multiple lines
\exhyphenpenalty 10000 % Prevent words from being broken over multiple lines

% Margins
\topmargin=-0.45in
\evensidemargin=0in
\oddsidemargin=0in
\textwidth=6.5in
\textheight=9.0in
\headsep=0.25in
\title{ \course \\\large Homework 8 }
\author{ Haadi Majeed }
\date{Spring 2022}


\pagestyle{fancy}
\fancyhead{}
\fancyfoot{}
\lhead{\course}
\chead{Haadi Majeed}
\rhead{Page \thepage}

\begin{document}
\maketitle
\pagebreak

% Optional TOC
%\tableofcontents
\pagebreak
\section{Problem 1}
 [15 points; 3 each]\\
Consider the following data set. There are two classes for
Y (mapped to \{-1, +1\}). There is one feature X that can be used for prediction.
\begin{center}
    \begin{tabular}{ |c|c| }
        \hline
        Y  & X  \\
        \hline
        -1 & 6  \\
        \hline
        -1 & 7  \\
        \hline
        -1 & 11 \\
        \hline
        +1 & 9  \\
        \hline
        +1 & 13 \\
        \hline
        +1 & 14 \\
        \hline
    \end{tabular}
\end{center}
\subsection{A}
Draw the scatter plot of the data by hand.\\

\subsection{B}
Sketch the prediction function $\hat{Y}(x)$ for the $k = 1$ nearest neighbour classifier on the scatter plot. Briefly explain your process.

\subsection{C}
Sketch the prediction function $\hat{Y}(x)$ for the $k = 3$ nearest neighbour classifier on the scatter plot. Briefly explain your process.

\subsection{D}
Sketch the prediction function $\hat{Y}(x)$ for the $k = 5$ nearest neighbour classifier on the scatter plot. Briefly explain your process.

\subsection{E}
Describe what would happen if you tried to use the max margin classifier for this data set.


\section{Problem 2}
 [20 points; 4 each A-E]\\
For this problem you will fit classifiers to the same data sets you used in the last homework.
\subsection{A}
Change the classifiers used to the following nearest neighbour classifiers:\\
\begin{itemize}
    \item 1-NN
    \item 10-NN
    \item 10-NN with distance-based weights (include argument weights='distance')
    \item Radius based neighbour classifiers for radii 3, 4, and 6 (uniform weights)
    \item Radius based neighbour classifiers for radii 3, 4, and 6 with distance-based weights
\end{itemize}

\subsection{B}









%--/Paper--
\end{document}